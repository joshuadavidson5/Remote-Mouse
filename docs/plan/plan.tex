\documentclass[12pt,a4paper,english,oneside]{article}

\usepackage[utf8]{inputenc}
\usepackage{ae,aecompl,url,natbib}
\usepackage[english]{babel}

\parskip 1ex
\parindent 0pt
\evensidemargin 0mm
\oddsidemargin 0mm
\textwidth 159.2mm
\topmargin 0mm
\headheight 0mm
\headsep 0mm
\textheight 246.2mm

\pagestyle{plain}

\bibpunct{(}{)}{;}{a}{,}{,}

% ---------------------------------------------------------------------

\begin{document}

\title{T-111.5360 Research Plan:\\[5mm]
Remote Mouse}

\author{Valter Kraemer 84669F \\
  Ville Skyttä 42818N \\
Aalto University}

\date{\today}

\maketitle

% ---------------------------------------------------------------------

\section{Introduction}

WebSockets is a technology providing an efficient full-duplex
bidirectional communications channel between clients and servers. Due
to its low latency characteristics, it is well suited for real time
applications. The WebSocket protocol is standardized by IETF
\citep{rfc} and W3C \citep{w3c} develops an API to enable its use in
web pages.

For the T-111.5360 course, we chose the WebSockets topic because we
found it interesting and different in nature compared to most other
modern web technologies. WebSockets is also a new technology for us in
the sense that we do not have prior hands-on experience with it. We
expect knowledge of it to be beneficial for us in both future studies
as well as in our current and future work.

Our practical research and implementation is going to be about a
virtual mouse pointer on a web page, controlled from another web
device. WebSockets will be used as the communications technology
between the devices.

\section{Related work}

State of the art in industry and research includes applications made
conceptually possible by the nature of WebSockets. Another prominent
area of research is exploration of WebSockets' performance
characteristics and applicability in performance sensitive
applications.

\citet{throughput} explores the practical boundaries of throughput
achievable with HTML5 applications. The study uses WebSockets and Web
Workers.

\citet{compression} compare use of several compression algorithms with
WebSockets. WebSockets itself is an optimization over using HTTP for
purposes it was not designed for, and utilizing compression with it
takes the optimization one step further for some applications.

\citet{http2} study the use of HTTP/2's server push functionality in
order to improve user experience by making video streaming start
faster in DASH. They use WebSockets for estimating bandwidth and
compare WebSockets over HTTP/1.1 using dedicated TCP connection to
WebSockets over HTTP/2 transport multiplexing.

\citet{bassbouss} address multi-screen web application development and
the transformation of traditional web applications to multi-screen
capabilities. Both the current and proposed multi-screen application
models utilize WebSocket in communications between clients (screens)
and servers.

\citet{projectrooms} present a system implementing a virtual project
room and real-time collaboration space to facilitate remote,
distributed teamwork. The implementation uses WebSockets for
inter-client communication and WebRTC signaling.

\citet{homeappliances} introduce an architecture for remote control for
home appliances. They address issues related to large scale deployment
of real time systems such as scalability and operational
costs. WebSockets is the central technology in their implementation.

Trello (\emph{http://www.trello.com}) is a web-based project
management application that uses WebSockets for realtime updating its
content, like tasks, statuses and assignees. They also use web polling
as a fallback if the browser doesn't support WebSockets
\citep{trello}.

\section{Research idea}

The application we are going to implement is a virtual mouse pointer
for websites that can be controlled from another web enabled device,
typically a mobile phone or a tablet. Communication between the remote
device and the host device will be handled by WebSockets. The commands
will be sent between the devices thru a server running Node.js with
Express.js on top.

Our goal is to first have simple click based navigation that works on
a test site. Later we want to be able to use the remote mouse on any
web page using a JavaScript bookmark. However, the method of using
JavaScript bookmarks won't support page transitions so it can only be
used at one page at a time and on single-page web applications. Later
we also want to opt-out the basic click based navigation for touch
based navigation.


\section{Working hours allocation}

\begin{tabular}{|p{120mm}|p{30mm}|}
  \hline
  Research plan                       & 8h    \\ \hline
  Working basic implementation        & 48h   \\ \hline
  State of the art                    & 10h   \\ \hline
  Midterm Demo                        & 4h    \\ \hline
  Fluid motion pointer movement       & 16h   \\ \hline
  Work on any single page application & 16h   \\ \hline
  Workshop Presentation               & 4h    \\ \hline
  Final Report                        & 32h   \\ \hline
\end{tabular}

\section{Schedule}

\begin{tabular}{|p{30mm}|p{120mm}|}
  \hline
  13.-29.10.2015      & Working basic implementation \\ \hline
  20.-27.10.2015      & State of the art \\ \hline
  3.11.2015           & Mid-term demo \\ \hline
  3.-17.11.2015       & Fluid motion pointer movement \\ \hline
  3.-17.11.2015       & Work on any single page application \\ \hline
  17.11.-3.12.2015    & Bug fixes and finishing touches \\ \hline
  8.12.2015           & Project workshop \\ \hline
  17.11-?.12.2015     & Final Report \\ \hline
\end{tabular}

% ---------------------------------------------------------------------

\bibliographystyle{aaltosci_t}

\bibliography{references}

% ---------------------------------------------------------------------

\end{document}
