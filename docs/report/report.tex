\documentclass[12pt,a4paper,english,oneside]{article}

\usepackage[utf8]{inputenc}
\usepackage{ae,aecompl,url,natbib,multicol,graphicx}
\usepackage[english]{babel}

\parskip 1ex
\parindent 0pt
\evensidemargin 0mm
\oddsidemargin 0mm
\textwidth 159.2mm
\topmargin 0mm
\headheight 0mm
\headsep 0mm
\textheight 246.2mm

\pagestyle{plain}

\bibpunct{(}{)}{;}{a}{,}{,}

% ---------------------------------------------------------------------

\begin{document}

\title{T-111.5360 Report:\\[5mm]
Remote Mouse}

\author{Valter Kraemer 84669F \\
  Ville Skyttä 42818N \\
Aalto University}

\date{\today}

\maketitle

\begin{multicols}{2}

% ---------------------------------------------------------------------

\section{Introduction}

WebSockets is a technology providing an efficient full-duplex
bidirectional communications channel between clients and servers. Due
to its low latency characteristics, it is well suited for real time
applications. The WebSocket protocol is standardized by IETF
\citep{rfc} and W3C \citep{w3c} develops an API to enable its use in
web pages.

In this report we introduce Remote Mouse, a virtual mouse pointer on a
web page, controlled from another web device. WebSockets is used as
the communications technology between the devices, through an
intermediate server.

[TODO: more]

\section{Related work}

\citet{bassbouss} address multi-screen web application development and
the transformation of traditional web applications to multi-screen
capabilities. Both the current and proposed multi-screen application
models utilize WebSocket in communications between clients (screens)
and servers. [TODO: more about this and how it is related to our stuff]

Agar.io\footnote{\url{http://agar.io/}} is a massively multiplayer
online game. In a nutshell, players control cells in a petri dish,
attempting to grow larger by consuming pellets and other cells, and
avoiding being consumed by other cells. The game is available for web
browsers on its website, and Android and iOS versions are available
for mobile devices. The web version uses an HTML canvas and its 2D
context for rendering, as well as HTML animation frames. Communication
between the browser and the server is implemented using
WebSockets. Data is transferred using WebSocket binary frames
(WebSocket opcode 2), which are constructed using the ECMAScript 6
ArrayBuffer and DataView objects. Due to the binary nature of the
data, the exact semantics of it are not available. During gameplay,
the traffic consists of on the order of 50 WebSocket frames per
second, with their sizes ranging approximately from a few to 200
bytes.

YouTube has a feature with which it is possible to control another
YouTube window's video controls from another screen, such as a
computer or a mobile device. It is primarly intended for controlling
YouTube on smart TVs, but can also be used in a browser. The
controlled screen\footnote{\url{http://www.youtube.com/tv}} is
operated by its paired
remote\footnote{\url{http://www.youtube.com/pair}}. The remote works
by sending POST requests to the server that forwards them to the
controlled window. It also uses polling every 10 seconds to check that
the controlled device is still available. The controlled window uses
long polling to check if any information is updated. YouTube is using
LocalStorage for storing information about the playback device and
different identifiers. MediaSource is used to attach sources to their
video elements.

Remot.io\footnote{\url{http://remot.io/}} is a service that controls
HTML presentations such as
reveal.js\footnote{\url{http://lab.hakim.se/reveal-js}} from touch
based devices. The controlling device sends POST requests to the
server that forwards them to the controlled device using long polling
by the controlled device. Remote.io is using touch events for their
remote. Swipe gestures translate to the directions the user wants to
navigate in the slides.

\section{Results}

[TODO]

\newcommand*\rot{\rotatebox{90}}

\begin{center}
  \begin{tabular}{rlrlrl}
    \rot{Latency} & \rot{Grade} & \rot{Latency} & \rot{Grade} & \rot{Latency} & \rot{Grade} \\
    \multicolumn{2}{c}{User 1} & \multicolumn{2}{c}{User 2} & \multicolumn{2}{c}{User 3} \\
    \hline
    500 & 2  & 400 & 2  & 0 & 8   \\
    200 & 4  & 100 & 5  & 500 & 1 \\
    100 & 5  & 300 & 4  & 300 & 4 \\
    0 & 6    & 0 & 8    & 100 & 8 \\
    100 & 6  & 100 & 8  & 0 & 9   \\
    0 & 8    & 0 & 8    & 200 & 7 \\
    \hline
  \end{tabular}
\end{center}

\begin{center}
  \begin{tabular}{rll}
    Network & Server & Latency \\
    \hline
    2G & Heroku & 500 \\
    3G & Heroku & 70 \\
    LTE & Heroku & 70 \\
    Wi-Fi & Heroku & 70 \\
    Wi-Fi & local & 5 \\
    \hline
  \end{tabular}
\end{center}

\section{Analysis}

[TODO]

\section{Conclusions}

[TODO]

% ---------------------------------------------------------------------

\bibliographystyle{aaltosci_t}

\bibliography{references}

% ---------------------------------------------------------------------

\end{multicols}

\end{document}
