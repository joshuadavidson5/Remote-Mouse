\documentclass[a4paper,english,twocolumn]{article}

\usepackage[utf8]{inputenc}
\usepackage{ae,aecompl,url,natbib,graphicx}
\usepackage[english]{babel}

\parskip 1ex
\parindent 0pt
\evensidemargin 0mm
\oddsidemargin 0mm
\textwidth 159.2mm
\topmargin 0mm
\headheight 0mm
\headsep 0mm
\textheight 246.2mm

\pagestyle{plain}

\bibpunct{(}{)}{;}{a}{,}{,}

\newcommand*\rot{\rotatebox{90}}

% ---------------------------------------------------------------------

\begin{document}

\title{T-111.5360 Report:\\[5mm]
Remote Mouse}

\author{Valter Kraemer 84669F \\
  Ville Skyttä 42818N \\
Aalto University}

\date{\today}

\maketitle

% ---------------------------------------------------------------------

\section{Introduction}

WebSockets is a technology providing an efficient full-duplex
bidirectional communications channel between clients and servers. Due
to its low latency characteristics, it is well suited for real time
applications. The WebSocket protocol is standardized by IETF
\citep{rfc} and W3C \citep{w3c} develops an API to enable its use in
web pages.

In this report we introduce Remote Mouse, a virtual mouse pointer on a
web page, controlled from another web device. WebSockets is used as
the communications technology between the devices, through an
intermediate server.

[TODO: more]

\section{Related work}

\citet{bassbouss} address multi-screen web application development and
the transformation of traditional web applications to multi-screen
capabilities. Both the current and proposed multi-screen application
models utilize WebSocket in communications between clients (screens)
and servers. [TODO: more about this and how it is related to our stuff]

Agar.io\footnote{\url{http://agar.io/}} is a massively multiplayer
online game. In a nutshell, players control cells in a petri dish,
attempting to grow larger by consuming pellets and other cells, and
avoiding being consumed by other cells. The game is available for web
browsers on its website, and Android and iOS versions are available
for mobile devices. The web version uses an HTML canvas and its 2D
context for rendering, as well as HTML animation frames. Communication
between the browser and the server is implemented using
WebSockets. Data is transferred using WebSocket binary frames
(WebSocket opcode 2), which are constructed using the ECMAScript 6
ArrayBuffer and DataView objects. Due to the binary nature of the
data, the exact semantics of it are not available. During gameplay,
the traffic consists of on the order of 50 WebSocket frames per
second, with their sizes ranging approximately from a few to 200
bytes.

YouTube has a feature with which it is possible to control another
YouTube window's video controls from another screen, such as a
computer or a mobile device. It is primarly intended for controlling
YouTube on smart TVs, but can also be used in a browser. The
controlled screen\footnote{\url{http://www.youtube.com/tv}} is
operated by its paired
remote\footnote{\url{http://www.youtube.com/pair}}. The remote works
by sending POST requests to the server that forwards them to the
controlled window. It also uses polling every 10 seconds to check that
the controlled device is still available. The controlled window uses
long polling to check if any information is updated. YouTube is using
LocalStorage for storing information about the playback device and
different identifiers. MediaSource is used to attach sources to their
video elements.

Remot.io\footnote{\url{http://remot.io/}} is a service that controls
HTML presentations such as
reveal.js\footnote{\url{http://lab.hakim.se/reveal-js}} from touch
based devices. The controlling device sends POST requests to the
server that forwards them to the controlled device using long polling
by the controlled device. Remot.io is using touch events for their
remote. Swipe gestures translate to the directions the user wants to
navigate in the slides.

\section{Results}

The controller is a web page. Most of it is empty, serving as the
control area.

For non-touch devices, the document has mousemove and click event
handlers. Each mousemove event results in the pointer position being
sent to the server in a WebSocket message.

For touch devices, the document contains event listeners
for touchstart, touchmove, and touchend.

Positions are sent as pairs of floating point numbers between 0 and 1,
representing the relative left and top position offsets of the event
in percents, proportional to the dimensions of the controller
window. The controllee moves the virtual pointer according to these
values, by converting them to percentages for the virtual pointer's
\verb!left! and \verb!top! CSS properties.

WebSocket messages for click events contain only the information that
the controller requests a click; there is no position information
included with it. The controllee locates the element which is the
target of the click based on its own virtual pointer position, using
the Document.elementFromPoint CSSOM View Module
method\footnote{\url{http://www.w3.org/TR/cssom-view/}}.

\begin{table} \centering
  \begin{tabular}{rl}
    Event & Example message \\
    \hline
    Session request & \verb!create:KJK4B! \\
    Session response & \verb!roomcode:KJK4B! \\
    Position change & \verb!pos:0.7245,0.6241! \\
    Click & \verb!click:left! \\
    Scroll & \verb!scroll:-0.004555! \\
    Latency adjustment & \verb!setLatency:10! \\
    Latency request & \verb!ping:null! \\
    Latency response & \verb!pong! \\
    Controllee not present & \verb!error:noClient! \\
    \hline
  \end{tabular}
  \caption{WebSocket message payloads}
  \label{table:messages}
\end{table}

The application's development is hosted in a GitHub repository located
at \url{https://github.com/valterkraemer/Remote-Mouse}. The server can
be run locally with the command \verb!node app.js! in the project's
top level directory. By default, it listens on port 3000 and can be
accessed at \url{http://localhost:3000/}. Pushes to the GitHub
repository are automatically deployed in the Heroku instance at
\url{https://remote-mouse.herokuapp.com/}. The Heroku instance uses
the secure \verb!https! and \verb!wss! protocols in order to make it
possible to control web applications served over both \verb!http! and
\verb!https! protocols with it.

\section{Analysis}

[TODO]

The most important technology for both controller and controllee sides
of the Remote Mouse implementation is WebSockets. According to
caniuse.com, it is fully supported in all current major browsers since
2013\footnote{\url{http://caniuse.com/#feat=websockets}}. The Node.js
ws library used on the server side has had releases available from
GitHub since
2011\footnote{\url{https://github.com/websockets/ws/releases}}.

The DeviceOrientation API used to implement scrolling based on
controller orientation is also well supported in current browsers to
the extent required by Remote Mouse. According to
caniuse.com\footnote{\url{http://caniuse.com/#feat=deviceorientation}},
only Microsoft Edge has full support for it, most other browsers have
partial support, and the desktop version of Safari has none. Safari's
non-support is not a major problem, because controller devices are
expected to be mobile ones, and the iOS Safari supports the API.

Table~\ref{table:caniuse} summarizes support for WebSockets and
DeviceOrientetation in major browsers. The versions listed are the
first ones in which full support for the technology appeared, followed
by the release year in parenthesis. If full support is not yet
available, the version number in parenthesis indicates the first
version with partial support.

\begin{table} \centering
  \begin{tabular}{rcc}
    & \rot{WebSockets} & \rot{DeviceOrientation} \\
    \hline
    IE & 10 (2012) & (11) (2013) \\
    Edge & 12 (2015) & (12) (2015) \\
    Firefox & 11 (2012) & (6) (2011) \\
    Chrome & 16 (2011) & (7) (2010) \\
    Safari & 7 (2013) & - \\
    iOS Safari & 6.1 (2013) & (4.3) (2011) \\
    Android Browser & 4.4 (2013) & (3) (2011) \\
    Chrome for Android & 47 (2015) & (47) (2015) \\
    \hline
  \end{tabular}
  \caption{caniuse.com: WebSockets and DeviceOrientation support in selected browsers}
  \label{table:caniuse}
\end{table}

To test the subjective effect of latency on user experience, a test
with three users was conducted. The users were first asked to use an
Apple Magic
Trackpad\footnote{\url{https://en.wikipedia.org/wiki/Magic_Trackpad}}
to get a feeling of a local, low latency user experience. Then, they
were asked to use Remote Mouse with the latency throttle set to
varying settings. The latency settings were shuffled, i.e. not
presented in increasing or decreasing order in order to avoid users'
expectations affecting the results. Users were tasked to grade the
quality of the pointer control experience in scale from 0 to 10, with
grade 0 being the lowest one, equal to unusable, and 10 being equally
good as the Magic Trackpad.

\begin{table} \centering
  \begin{tabular}{rlrlrl}
    \rot{Latency} & \rot{Grade} & \rot{Latency} & \rot{Grade} & \rot{Latency} & \rot{Grade} \\
    \multicolumn{2}{c}{User 1} & \multicolumn{2}{c}{User 2} & \multicolumn{2}{c}{User 3} \\
    \hline
    500 ms & 2  & 400 ms & 2  & 0 ms   & 8 \\
    200 ms & 4  & 100 ms & 5  & 500 ms & 1 \\
    100 ms & 5  & 300 ms & 4  & 300 ms & 4 \\
      0 ms & 6    & 0 ms & 8  & 100 ms & 8 \\
    100 ms & 6  & 100 ms & 8  & 0 ms   & 9 \\
      0 ms & 8    & 0 ms & 8  & 200 ms & 7 \\
    \hline
  \end{tabular}
  \caption{Subjective user test results}
  \label{table:userresults}
\end{table}

All three users rated the experience to belong in the middle of the
scale at approximately 250 ms latency. 500 ms was classified as barely
usable, and 0 to 100 ms quite acceptable. User test data is included
in table~\ref{table:userresults}.

To aid in estimating how these estimates translate to use of Remote
Mouse in different network setups, table~\ref{table:setuplatencies}
lists the typical latencies when the service is running in Heroku and
locally, and when it is being used over different network
connections.

\begin{table} \centering
  \begin{tabular}{rll}
    Network & Server & Latency \\
    \hline
    2G & Heroku    & 500 ms \\
    3G & Heroku    & 70 ms \\
    LTE & Heroku   & 70 ms \\
    Wi-Fi & Heroku & 70 ms \\
    Wi-Fi & local  & 5 ms \\
    \hline
  \end{tabular}
  \caption{Typical setup latencies}
  \label{table:setuplatencies}
\end{table}

[TODO: something about bandwidth usage, single WS message per MTU?]

\section{Conclusions}

[TODO]

Latency of a network connection is much more important for
satisfactory user experience with Remote Mouse than its
bandwidth. Bandwidth needs of the application are already quite modest
with the current implementation, and could be further reduced, for
example by using a more efficient binary WebSocket message payloads,
and compression. However, given the already low requirements and
possibility of getting negative effects on latency from optimizing for
bandwidth usage, these possibilities were not pursued as they are not
likely to result in significant overall user experience improvements,
if any.

Based on the test conducted as well as the authors' own experiences,
the latency goal for acceptable Remote Mouse user experience should be
set to the 0 to 100 ms range. According to our test results, these
kinds of latencies can be achieved with 3G and better mobile network
connections; 2G connectivity is not sufficient.

The technology stack related to WebSockets is stable and ready for
production use in both browser and server side. We did not run into
any issues on either browser or server side during the Remote Mouse
development process that would have been related to WebSockets
implementations. On the contrary, we found the APIs and
implementations very easy to use, and their performance matches or
exceeds the requirements for Remote Mouse.

A prominent use case for Remote Mouse is remote control of web based
presentations and applications, using for example a laptop computer to
host the presentation or web application and displaying its screen to
viewers, while controlling it remotely from a mobile device. Because
the contents of the controllee screen are not visible in the
controller, this use case is in our implementation limited to setups
where the user operating the controller can see the controllee
screen. If the controllee screen would be available for remote
viewing, use cases like for example remote assistance of web
application use would be quite relevant. The technology and principle
of tracking the pointer or touch movements could also be used for
recording user actions on a web site, for example for usability
evaluation, user interface research, and trials.

% ---------------------------------------------------------------------

\bibliographystyle{aaltosci_t}

\bibliography{references}

% ---------------------------------------------------------------------

\end{document}
